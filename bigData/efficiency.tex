\subsection{Efficiency based on the Number of Bins}
\label{sect:effstudy}
In this section, we address the number of bins in terms of asymptotic efficiency.
This study indicates that 100 bins seem to suffice for moderate sample sizes to millions of observations.
We set the default to \verb|B=1000| in \verb|bigRfit| based on empirical evaluation of simulated datasets using ranges of values of $B$ considering computational time and relative efficiency.\footnote{Computational time was similar when using 100 and 1000 bins, for example, on a mid-range laptop.}

Let $\varphi(u)$ be the selected score function for estimation and associated inference.
Intuitively, the step and linear score functions which approximate $\varphi(u)$
converge to $\varphi(u)$ as the number of bins goes to $\infty$
(see \citet{bigRfitpaper} for more information).
%We briefly sketch a formal argument of this fact
%for the step scores.
%A similar argument holds for the piecewise linear scores.

%Because $\varphi(u)$ is bounded and continuous on $[0,1]$, it is uniformly continuous on $[0,1]$.
%Write the step score approximation function as $\varphi_B(u)$ where $B$ is the number of bins.
%Let $\epsilon >0$ be given and choose $\delta >0$ such that
%$|\varphi(u_1)-\varphi(u_2)|< \epsilon$, if $|u_1-u_2|<\delta$.
%Choose $B_0 > 1/\delta$.
%Let  $B>B_0$.
%Let $u \in [0,1]$.
%Then $u$ is in some subinterval of the partition using $B$; i.e., for some $i$, $(i-1)/B \leq u < i/B$.
%Hence $\varphi[(i-1)/B] \leq \varphi_B[u] \leq \varphi[i/B]$.
%Then, since $\varphi(u)$ is nondecreasing and $1/B < \delta$, we have
%\begin{equation}
%\label{cal1}
%|\varphi_B(u) - \varphi(u)| \leq \left|\varphi\left(\frac{i}{B}\right) - \varphi\left(\frac{i-1}{B}\right) \right| < \epsilon.
%\end{equation}
%Hence, we have the theorem:

%\begin{theorem}
	%Let $\varphi_B(u)$ be the approximating step (linear) score for $\varphi(u)$.
	%Then $\varphi_B(u)$ converges uniformly to $\varphi(u)$ on $[0,1]$.
%\end{theorem}
%
%Using the inequality $|a| - |b| \leq |a-b|$ and letting $\epsilon =1$, we have for $b$ sufficiently large
%\begin{equation}
%\label{cal2}
	%|\varphi_B(u)| \leq |\varphi(u)| + 1.
%\end{equation}
%
%\begin{theorem}
%The efficacy of the piecewise linear and the step scores converge to the efficacy of approximated score
%as the number of bins converges to $\infty$.
%\end{theorem}

%\noindent
%\emph{Proof:} 
%First note that because $\varphi(u)$ and $\varphi_f(u)$ are both square integrable on $(0,1)$ both functions are 
%integrable
%and, further, by the Cauchy-Schwarz inequality, their product is integrable.
%By (\ref{cal2}), we have
%\begin{equation}
	%\label{cal3}
	%|\varphi_B(u)\varphi_f(u) | \leq (|\varphi(u)|+1)|\varphi_f(u)|,
%\end{equation}
%for $B$ sufficiently large.
%The product of functions on the right side of the inequality (\ref{cal3}) is integrable.
%Hence, since $\varphi_B(u) \rightarrow \varphi(u)$, as $B \rightarrow \infty$, we have
%by the Lebesgue Dominated Convergence Theorem
%\begin{equation}
        %\label{ssefficacy}
%\lim_{B\rightarrow \infty} \int_0^1 \varphi_B(u) \varphi_f(u) \, du =
 %\int_0^1 \lim_{B\rightarrow \infty} \varphi_B(u) \varphi_f(u) \, du =
 %\int_0^1 \varphi(u) \varphi_f(u) \, du;.
%\end{equation}
%that is,  $\gamma_{B} \rightarrow \gamma_{\varphi}$ as the number of bins goes to $\infty$.
%\rule{1.8mm}{1.8mm}
%
%This result is general.
In the next two subsections, we obtain the efficacy of the step scores which approximate
the Wilcoxon scores for normal and logistic distributed errors, respectively.
%For these two situations, $B=100$ seems to be a sufficient number of bins.

\subsubsection{Efficiencies of the Step Scores at the Normal Distribution}
\label{asyeffnorm}
We determine the efficiencies of the step scores approximating the
the Wilcoxon score function ($\varphi(u) = \sqrt{12}[u-(1/2)]$) at the normal distribution.
As discussed previously in the text, the ARE of the Wilcoxon
estimator relative to the LS estimator is 0.9549297.
%When fitting a dataset using the big data algorithm,
%the bin lengths differ.
%For this investigation, though, we consider the ideal condition where the bin lengths are the same.
%From expression  (\ref{efficacy}), the efficacy depends on the distribution of the random errors but, since it is the reciprocal of a scale
%parameter it is invariant to location and varies inversely with respect to scale.
%Thus, we need only consider the form of the pdf.
%So for the normal case we select the standard normal pdf.

%For this case the function $\varphi_f(u)$ in expression (\ref{optscore}) simplifies to
%\begin{equation}
%\label{phi_f}
%\varphi_f(u) = \Phi^{-1}(u),
%\end{equation}
%where $\Phi(x)$ is the standard normal cdf; i.e., the optimal scores at the normal distribution are these scores,
%which are called the {normal scores}.
%
%The efficacy for the Wilcoxon case is the well known result:
%\begin{equation}
%\label{effwil}
%\gamma_W =  \int_0^1 \sqrt{12}[u-(1/2)]\Phi^{-1}(u) \, du = \sqrt{\frac{\pi}{3}} = 1.023327.
%\end{equation}
%Note that square of the reciprocal of $\sqrt{\frac{\pi}{3}}$ is the ARE of the Wilcoxon
%estimator relative to the LS estimator, 0.9549297.

%Consider a step score function, $\varphi_S$, as discussed in Section \ref{stpalg},  over $B$ bins; i.e., over the intervals
%\begin{equation}
%\label{stepints}
%\left[\frac{i-1}{B}, \frac{i}{B}\right), \;\;\; i=1, 2, \ldots, B.
%\end{equation}
%The function $\varphi_S(u)$ is constant on each of these intervals and the size of the step is the same.
%Let $s_1<s_2 < \cdots <s_B$ be the respective values of $\varphi_S(u)$ on the $B$ subintervals.
%Assume that it is standardized, i.e., expression (\ref{eq:score1}) holds.
%Then the efficacy of $\varphi_B(u)$ is
%\begin{eqnarray}
%	\gamma_{\varphi_B} &=& \int_0^1 \varphi_{\varphi_B}(u) \Phi^{-1}(u) \, du  \nonumber \\
% &=& \sum_{i=1}^B \int_{(i-1)/B}^{i/B}  s_i \Phi^{-1}(u) \, du  \nonumber \\
% &=& \sum_{i=1}^B s_i \int_{\Phi^{-1}[(i-1)/B]}^{\Phi^{-1}[i/B]}  x \frac{1}{\sqrt{2\pi}} e^{-x^2/2} \, dx  \nonumber \\
% &=& \sum_{i=1}^B \frac{s_i}{\sqrt{2\pi}}  \int_{(1/2)\Phi^{-1}[(i-1)/B]^2}^{(1/2)\Phi^{-1}[i/B]^2} e^{-z} \, dz  \nonumber \\
%\label{effvarS}
% &=& \sum_{i=1}^B \frac{s_i}{\sqrt{2\pi}} \left[\exp\left\{-(1/2)\Phi^{-1}[(i-1)/B]^2\right\}
%            - \exp\left\{-(1/2)\Phi^{-1}[i/B]^2\right\} \right],
%\end{eqnarray}
%where the derivation included two changes-of-variables.

%The scores are computed as in Section \ref{}.
%Expression (\ref{effvarS}), though,  is asymptotic.
For illustration, we use a sample size of $N=10000$. 
Table \ref{tabeff} displays the calculated efficiencies of the step scores for various values of $B$.
It also tables their ARE's relative to the Wilcoxon and to LS.
Of course, LS dominates since the error distribution is normal.
Note that the step scores for $B=2$ are the sign scores; with ARE relative to LS of 
0.637.
At $B=100$ there is little difference between the step scores and the Wilcoxon scores in terms
of efficiency. 

\begin{table}[ht]
\centering
\begin{tabular}{r|rr|rr}
  \hline
 & \multicolumn{2}{c|}{Efficiency} & \multicolumn{2}{|c}{ARE} \\
 B & SS & W & SS to W & SS to LS \\ 
  \hline
 2 & 0.797885 & 0.977205 & 0.666667 & 0.636620 \\ 
 4 & 0.925281 & 0.977205 & 0.896553 & 0.856145 \\ 
 5 & 0.942297 & 0.977205 & 0.929831 & 0.887923 \\ 
 10 & 0.967152 & 0.977205 & 0.979531 & 0.935383 \\ 
 15 & 0.972493 & 0.977205 & 0.990380 & 0.945743 \\ 
 20 & 0.974368 & 0.977205 & 0.994203 & 0.949394 \\ 
 50 & 0.976687 & 0.977205 & 0.998939 & 0.953917 \\ 
 100 & 0.977064 & 0.977205 & 0.999711 & 0.954654 \\ 
 200 & 0.977167 & 0.977205 & 0.999922 & 0.954855 \\ 
 400 & 0.977195 & 0.977205 & 0.999979 & 0.954910 \\ 
   \hline
\end{tabular}
\caption{
Table of efficiencies of step scores for normal errors.
Abbreviations: ARE=asymptotic relative efficiency; B is number of bins; SS=step scores; W=Wilcoxon scores (based on a full ranking); LS=least squares.
}
\label{tabeff}
\end{table}
\subsubsection{Wilcoxon Scores at the Logistic Distribution}
In the last subsection, we assumed a normal population while in this section we consider a logistic
population.
Recall that the Wilcoxon scores are optimal in this case.
The efficacy for the Wilcoxon score function $\varphi_W(u) = \sqrt{12}[u-(1/2)]$ at the logistic
probability model is 0.5773503
%\begin{equation}
%\label{wilefflog}
%\gamma_W = \int_0^1  \sqrt{12}\left(u - \frac{1}{2}\right)(2u-1)\, du  = \frac{2}{\sqrt{12}} \dot= 0.5773503.
%\end{equation}

%The pdf and cdf of the logistic distribution are given by
%\begin{equation}
%\label{logisticpc}
%\mbox{$f(x) = \frac{e^{-x}}{(1+e^{-x})^2}\;\;$ and $\;\;F(x) = \frac{1}{1+e^{-x}}$, $\; -\infty < x < \infty$},
%\end{equation}
%respectively.
%It follows that $F^{-1}(u) = \log[u/(1-u)]$.
%By a simple derivation, the optimal scores are given by
%\begin{equation}
%\label{optlogistic}
%\varphi_f(u) = 2u -1;
%\end{equation}
%i.e., Wilcoxon scores.
%Hence, the efficacy for the Wilcoxon score function $\varphi_W(u) = \sqrt{12}[u-(1/2)]$ at the logistic
%probability model (\ref{logisticpc}) is:
%\begin{equation}
%\label{wilefflog}
%\gamma_W = \int_0^1  \sqrt{12}\left(u - \frac{1}{2}\right)(2u-1)\, du  = \frac{2}{\sqrt{12}} \dot= 0.5773503.
%\end{equation}
%As the following derivation shows, we also obtain the efficacy of the approximating step scores in a simplified form:
%\begin{equation}
%\label{effsteplogistic}
%\gamma_s = \sum_{i=1}^B \int_{(i-1)/B}^{i/B} s_i (2u-1)\, du
% = \sum_{i=1}^B s_i \frac{2i-(B+1)}{B^2}.
%\end{equation}

Table~\ref{tabefflog} displays the  efficiencies of step scores for various values of B relative to the efficacy of the Wilcoxon at the logistic distribution. 
At $B=2$ the step scores are the sign scores and, of course, they are inefficient relative to the Wilcoxon scores in the case of logistic errors.
Note, though, at $B=5$ this ARE has climbed to 0.96.
At $B=100$ the efficiency of step score analyses is essentially the same as the Wilcoxon analyses.
Also at $B=4$, the step scores analyses are more efficient than LS analyses.

\begin{table}[ht]
\centering
\begin{tabular}{r|rr|rr}
  \hline
 & \multicolumn{2}{c|}{Efficiency} & \multicolumn{2}{|c}{ARE} \\
B & SS & W & SS to W & SS to LS \\ 
  \hline
 2 & 0.500000 & 0.577350 & 0.750000 & 0.822467 \\ 
 4 & 0.559017 & 0.577350 & 0.937500 & 1.028084 \\ 
 5 & 0.565685 & 0.577350 & 0.960000 & 1.052758 \\ 
 10 & 0.574456 & 0.577350 & 0.990000 & 1.085656 \\ 
 15 & 0.576122 & 0.577350 & 0.995751 & 1.091963 \\ 
 20 & 0.576628 & 0.577350 & 0.997500 & 1.093881 \\ 
 50 & 0.577235 & 0.577350 & 0.999600 & 1.096184 \\ 
 100 & 0.577321 & 0.577350 & 0.999900 & 1.096513 \\ 
 200 & 0.577343 & 0.577350 & 0.999975 & 1.096595 \\ 
 400 & 0.577348 & 0.577350 & 0.999994 & 1.096616 \\ 
   \hline
\end{tabular}
\caption{
Table of efficiencies of step scores for logistic errors.
Abbreviations: ARE=asymptotic relative efficiency; B is number of bins; SS=step scores; W=Wilcoxon scores (based on a full ranking); LS=least squares.
}
\label{tabefflog}
\end{table}

