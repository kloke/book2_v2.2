\subsection{Big Rfit Algorithm}
\label{sect:BigRfit}
%{\bf NEEDS EDIT}
%Our current big data algorithm  for rank-based fits and analyses is  called \verb|BigRfit|.
The algorithm for computing the rank-based estimates is an iterative Newton-type algorithm (see \citet*{bigRfitpaper}; \citet{hm11}).

Implementation of developmental (at the time of writing) versions are discussed further at: 
 \verb|github.com/kloke/bigRfit|.\index{R package!{\tt bigRfit}}
Currently, we consider the primary interface to be \verb|bigRreg|.
The package
\texttt{data.table} \citep{data.table}
is used for binning the residuals once the \verb|cut|s have been made as well as scores calculations. 
This algorithm is illustrated in Subsection~\ref{SS:data.table}
Main concepts of the \verb|bigRreg| interface are outlined in Section~\ref{S:bigRreg}.
A one-step estimate using 
\texttt{sparklyr} \citep*{sparklyr}
is outlined in Section~\ref{S:1stepWsparklyr}.
%The default score function is the Wilcoxon linear score function which results in an analysis with 95\% efficiency relative to the LS analysis when the random errors of the model have a normal distribution.
%Other score functions, though, can be selected including those in the package \verb|Rfit|.

The software, \verb|bigRreg|, employs the step score function corresponding to the score function selected.
In the next section, we discuss the number of bins in terms of efficiency which indicates about 100 bins is sufficient, however,
with massive data sets in mind, we have set the default number of bins at 1000.
For the associated analysis (e.g., 95\% confidence intervals), the software uses the estimates of the scale parameters provided in expressions (\ref{E:bingamma}) and (\ref{bdtauS}).

%We are developing an R package \texttt{bigRfit} for this algorithm.
%A developmental version of the package is available at \texttt{https://github.com/kloke/bigRfit}.
%In the short time study presented in Section \ref{sect:ts}, for the big data situations its computation time is in the ``ballpark'' of the time for LS computation.
%We are also in the process of developing a \texttt{sparklyr} \citep*{sparklyr} implementation
%which makes use of the \texttt{dplyr} \citep*{dplyr} interface to the spark dataframes.
